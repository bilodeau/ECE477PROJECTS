\section{Flight Dynamics}
\paragraph{}
Throughout this paper we us a number of terms to describe quadrotor flight dynamics.  We begin with a brief discussion and some definitions.  A quadrotor has six degrees of freedom.  Two in the horizontal plane (forward/back and left/right), one vertically (altitude), and 3 axes of rotation.  We decided to define one of the four rotors as the 'front' of the craft.  This rotor we call 'north'; the others are 'south,' 'east,' and 'west' respectively.  Some designs define a pair of rotors as the front.  This is known as flying 'X' mode as opposed to '+' mode like our project.  Using standard flight terminology, we defined three axes of rotation as follows: rotation about the vertical axis is called yaw, where counter-clockwise (turning left) is the positive direction, rotation about the horizontal axis perpendicular to the forward direction is called pitch, north rotor up (leaning back) is the positive direction, finally, rotation about the horizontal axis parallel to the forward direction is called roll, where leaning left is the positive direction.
\paragraph{}
Adjusting the roll and pitch of the craft in flight is fairly straightforward.  If one rotor is lower than the others, then the quadrotor is leaning in that direction.  We can increase the thrust of that motor to compensate and level out.  Yaw, on the other hand, is slightly more complicated.  We use 2 pair of propellers rotating in opposite directions.  Each pair, then, generates a torque opposing that of the other pair.  We can rotate in the yaw direction by speeding up one set of rotors with respect to the other.  Since simply speeding up two of the motors will generate more overall lift, the quadrotor will fly up.  Thus we need to reduce the thrust of the opposite pair of rotors any time as we increase the thrust of a given pair.  This will keep the overall thrust constant, while still producing a yaw rotation.  Similarly, when correcting roll and pitch, we decrease the thrust of the higher rotor at the same time we increase the thrust of the lower one.  What the controls really adjust then, is the relative speed of opposing rotors or pairs of rotors.
