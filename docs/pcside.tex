\section{PC side}
\paragraph{}
The PC side setup which we use to communicate with the quadrotor consists of two parts. First we have a Sheaf-Monk utility board connected to an XBee wireless modules.  This lets us open a serial port to send and receive data wirelessly.  Secondly, we have a PC console application which displays the realtime sensor data readings and saves serial communication messages to a log file.  The console application also allows the user to type in and send ASCII based commands to the quadrotor.  Typical usage would be something like:\\
\newline
\begin{lstlisting}[frame=single]
BOOT
IDLE // checks motor operations and communications
STOP
BEGIN // enable balancing and altitude controllers
TAKEOFF // enable takeoff controller
LAND
STOP 
\end{lstlisting}

\clearpage
The PC side application is run from the command line and typically appears as illustrated below:\\
\begin{lstlisting}[frame=single]
Last Command Received:                  Hex:

Compass Heading (degrees): 0.0          Sonar Distance (feet): 0.0
Barometer Temperature: 0.0              Barometer Pressure: 0.0
Barometer Altitude: 0.0
Nunchuck X: 0
Nunchuck Y: 0
Nunchuck Z: 0
Gyroscope Roll (deg/sec): 0             Gyroscope Pitch (deg/sec): 0
Yaw   (deg): 0                          Alt Gain: 0
Pitch (deg): 0                          Pitch Gain: 0
Roll  (deg): 0                          Roll Gain: 0








Type 'q' to quit.
:
Number of bytes sent:  0
Last Command Sent:                      Hex:
\end{lstlisting}
