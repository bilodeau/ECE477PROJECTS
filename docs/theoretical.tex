
%Section Discusses Control
\section{PID Controllers}


The quadrotor uses different PID controllers to control flight. PID controllers were selected for easy tuning and effective control. 

\subsection{PID Theory}
Equation \ref{PIDgen} gives the generic equation for any PID controller. 

\begin{equation}
D(s) = K_p+\frac{K_i}{s}+K_ds
\label{PIDgen}
\end{equation}

K$_p$, K$_i$, and K$_d$ are the proportional, integral, and derivative constants.  Each term manipulates the error from the desired output and applies a new input signal to the quadrotor. 

\subsubsection{Proportional}
The proportional constant takes the error from the plant and inputs a new value proportional to the error. A PID controller is not implementable without a proportional constant.  This term guides the controller to the proper final value. 

\subsubsection{Derivative}
The derivative constant dampens the error to avoid oscillation. 

\subsubsection{Integral}
The integral part of the PID controller reduces the effect of outside forces on the quadrotor.  So if the quadrotor nears gusty air, theoretically this term would help compensate for the interference. 

\subsection{Controllers, Testing and Tuning}

The flight  PID control tunning applied the Ziegler-Nichols method as no knowledge of how to model the plant existed.
\subsubsection{Altitude Control}
\subsubsection{Yaw Control}
\subsubsection{Pitch Control}
\subsubsection{Roll Control}