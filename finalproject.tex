\documentclass{article}
\usepackage{graphicx}
\usepackage{url}
	\addtolength{\oddsidemargin}{-.875in}
	\addtolength{\evensidemargin}{-.875in}
	\addtolength{\textwidth}{1.75in}
	\addtolength{\topmargin}{-.875in}
\addtolength{\textheight}{1.75in}
\begin{document}
\title{ECE 477 - Final Project}
\author{Peter Bilodeau\\
	peterjbilodeau@gmail.com\and
	Brady Butler\\
	m.brady.butler@gmail.com \and
	Breanna Stanaway\\
	breanna.stanaway@gmail.com\and
	Cody Morgan\\
	cody.b.morgan@gmail.com}
\date{\today}
\maketitle{}
\section{ABOUT THIS DOCUMENT}
This is supposed to be the basis of our final lab report.  Ideally, each section of the report will be fleshed out as we work on the project.  This should be mostly complete by the time we have stable flight.  Right now I've just got a list of concerns under each heading.  Most of the things listed are either a question that I don't have an answer to yet, or a task / subtask that we still need to complete.  I'm thinking that we can include the answers to these questions in the final spec / description of the project, and we can describe each element fully as we complete each associated task.

I've listed part numbers (and sparkfun SKUs, they have spec sheets and examples) for each of the devices I have already (or if they are in the mail).  If something is obviously wrong, or there's a much better alternative out there let me know ASAP so I can get something on order.  I'm anticipating placing another parts order in the near future so please share any 'maybe' ideas also.  I definitely do not want to run into a time crunch with parts still in the mail.

\section{Frame}
The quadcopter frame is constructed from 3/4" square aluminium tubing, in a 'X' with 55cm arms.  Motors are mounted 3cm from the end of each arm, using 5mm bolts, washers and locknuts.  The center of the frame has matching notches in each arm and plastic squares are bolted above and below to provide an electronics mounting platform.

electronics mounting (sonar / sensors)\\
electonics protection\\
battery mount\\
battery protection\\
landing gear\\
ESC mounting\\
wire dressing\\

\section{Power}
11.8v spider (13 or 14 gauge?)\\
11.8v battery connectors (ZIPPY FLightmax 3000mh 3S1P 20C)\\
5-6v UBEC (HXT UBEC/3735 HXT UBEC 5/6v output, 5.5~23v Input) \\
3.3v power supply (voltage regulator)\\
Li-Po alarm (TURNIGY 3~8S Voltage Detector)\\
battery life (on the order of 3-10minutes?)\\
do we need to order another / bigger battey pack?\\
battery charging is a little complicated\\
battery storage (thing is dangerous man)\\

\section{Communication}
PC to ATmega328 via serial\\
wireless coms over ZigBee?\\

Serial communications are buffered on both the PC and AVR.  Communications sent and received by the PC are dumped into a log file for later reference.  (Logs should be timestamped...)

Possible Commands to implement:\\

SHUTDOWN // turns off all motors.  do we want a remote power off for the AVR or sensors?\\
KILL // annihilates the human race\\
IDLE // spins all rotors at the minimum 'turnover' rate to save power without shutting down\\
LAND // stops any horizontal motion and lands in the current location\\
HOVER xsome heightx  // climbs or descends to the given altitude and stabilizes\\
ROTATE xdegreesx    // spins the quadcopter around the yaw axis\\
DISCO BALL MODE      // not sure... this one was Brady's idea.\\
// some demo modes? various hover heights on a timer?  simple fly there and land?\\

\section{Motor Control}
I'm thinking we fly in '+' mode rather than 'X' mode.  We can label our motors as North, South, East, West, or maybe fore, aft, port, starboard?

The motors selected have a maximum thrust of 900+g.  With 10x6 propellers, expected thrust is about 700g each.

What are the thrust limits for these motors / props? Any vibration or balancing issues?\\
What is the function mapping RPM to thrust?  How much thrust do we lose when flying at an angle?\\
How much of an RPM difference is necessary to generate rotation about the yaw axis?\\

HK-SS18A/6456 Hobbyking SS Series 15-18A ESC\\
DT700/6246 hexTronik DT700 Brushless Outrunner 700kv Motors\\
HCB-10/11333 10x6 Propellers (Standard and Counter Rotating) These are 10 inch props.\\

ESC input spec (either PWM or pulse per second based?)\\
ESC 'programming' via full throttle/idle toggling on boot up\\
ESC modes are based on battery type?\\

\section{Microcontroller}
ATmega328 - the chip I have currently has an Arduino bootloader on it, not sure what that means with respect to fuses / programming

\section{Sensors}
what can we use for indoor (or short distance) position/movement sensing?

\subsection{Sonar}
PING))) Ultrasonic Distance Sensor\\
2 cm to 3m range\\
use for takeoff and landing as well as smart low altitude hold\\
already have some working AVR code for this device

\subsection{2-Axis Gyroscope}
IDG500 - SEN-09094\\
measures angular velocity\\
has 2 output pins for each axis, one high resolution, one low\\
need to filter output (probably in software) to adjust for vibration\\

\subsection{$I^2C$ Devices}
Logic Level Converter (BOB-08745)\\
The $I^2C$ devices below all run at 3.3v (except maybe the nunchuck?), but I do have a few of the 
converters listed above.  Probably could do it with just one converter if we have everything on one $I^2C$ bus.  I think $I^2C$ uses some sort of slave addressing protocol.

\subsubsection{Barometer}
BMP085 - SEN-09694\\
I think this device and the 2-Axis gyro both have some sort of temperature output which we should be able to use to correct for air pressure differences due to the outdoor temperature.  Hopefully we can get a reasonable altitude reading.

\subsubsection{3-Axis Magnetometer}
HMC5883L - SEN-10530\\
need to filter output (probably in software) to adjust for vibration\\
what's the magnetic declination here?\\
do we need to monitor all 3 axes or will just the yaw reading be enough?\\

\subsubsection{Accelerometer/Gyroscope}
Wii Nunchuck board (lots of hacking tutorials online)\\
pretty sure that the board takes care of all of the data combination / drift correction work for reading angular position, we just have to interface with it serial\\
need to filter output (probably in software) to adjust for vibration\\

\subsubsection{GPS}
MediaTek MT3329 GPS 10Hz (got this from DIYDrones.com.  I believe it has custom firmware at the moment and they've got some spec sheets up on the site.)\\
position data format\\
position data resolution\\
velocity data format\\
velocity data resolution\\
imput power source has to meet some particular specs (low ripple?) in order to guarantee that we get good data\\

\end{document}
